\section{Basic Concepts}\label{sec:basic-concept}

\subsection{Machine Learning Categories}
\label{subsec:machine-learning-categories}

\textbf{Machine learning (ML)} is a subset of \textbf{Artificial Intelligence (AI)} that enables computers to learn from data and improve their performance over time without being explicitly programmed.
That primary goal is to develop models that can generalize well to unseen data, making accurate predictions or decisions.
Types of machine learning consist of:

\begin{itemize}
    \item { \textbf{Supervised Learning}: The model learns from labeled data. }
    \item { \textbf{Unsupervised Learning}: The model finds patterns in unlabeled data. }
    \item { \textbf{Semi-supervised Learning}: Uses a small amount of labeled data with a large amount of unlabeled data. }
    \item { \textbf{Reinforcement Learning}: The model learns by interacting with an environment and receiving rewards or penalties. }
\end{itemize}

Machine learning problems are categorized based on the type of output they generate and the nature of the task:

\begin{itemize}
    \item {
        \textbf{Classification}: Assign data points to predefined categories (labels). The output is discrete and categorical.
        \begin{itemize}
            \item { Spam detection }
            \item { Image classification }
            \item { Disease diagnosis }
        \end{itemize}
    }
    \item {
        \textbf{Regression}: Predict continuous values based on input data.
        \begin{itemize}
            \item { Predicting house prices. }
            \item { Estimating stock prices. }
            \item { Forecasting temperature. }
        \end{itemize}
    }
    \item {
        \textbf{Clustering}: Group similar data points together without predefined labels.
        \begin{itemize}
            \item { Customer segmentation (grouping customers by shopping behavior). }
            \item { Document clustering (organizing articles by topics). }
            \item { Image segmentation. }
        \end{itemize}
    }
    \item {
        \textbf{Density Estimation}: Estimate the probability density function of a dataset. The output is a continuous function describing the data distribution.
        \begin{itemize}
            \item { Anomaly detection (fraud detection, network security). }
            \item { Data generation. }
            \item { Feature engineering (transforming features based on density). }
        \end{itemize}
    }
\end{itemize}
