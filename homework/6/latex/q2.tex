\question Q2\droppoints

\begin{solution}
    \text{(a)} A $2 \times 2$ Gram matrix $\mathbf{K}$ formed from two vectors $\mathbf{v}_1$ and $\mathbf{v}_2$ has the form:

    \[
        \mathbf{K} =
        \begin{bmatrix}
            \mathbf{v}_1 \cdot \mathbf{v}_1 & \mathbf{v}_1 \cdot \mathbf{v}_2 \\
            \mathbf{v}_2 \cdot \mathbf{v}_1 & \mathbf{v}_2 \cdot \mathbf{v}_2 \\
        \end{bmatrix}
        =
        \begin{bmatrix}
            \| \mathbf{v}_1 \|^2      & \mathbf{v}_1 \mathbf{v}_2 \\
            \mathbf{v}_2 \mathbf{v}_1 & \| \mathbf{v}_2 \|^2      \\
        \end{bmatrix}
    \]

    Therefore, the determinant of $\mathbf{K}$ is:

    \[
        \det(\mathbf{K}) = \| \mathbf{v}_1 \|^2 + \| \mathbf{v}_2 \|^2 - 2\mathbf{v}_1 \mathbf{v}_2
    \]

    \text{Before (b)} We will show that if all the eigenvalues of a matrix $M$ are positive, then the determinant of $M$ is positive.
    The characteristic equation of a matrix $M$ is given by:

    \[
        \det(M - \lambda I) = 0
    \]

    Solving this equation yields the eigenvalues of the matrix $M$, which means:

    \[
        \det(M - \lambda I) = \prod_{i = 1}^{n}(\lambda_i - \lambda)
    \]

    Let $\lambda = 0$, we have:

    \[
        \det(M) = \prod_{i = 1}^{n}(\lambda_i) = \lambda_1 \lambda_2 \dots \lambda_n
    \]

    Since all the eigenvalues of $M$ are positive, we can conclude that:

    \[
        \det(M) > 0
    \]

    \text{(b)}
    Let's define a $2 \times 2$ Gram matrix $\mathbf{K}$ constructed from a positive definite kernel $k(x, x')$, evaluated at two points $x_1$ and $x_2$:

    \[
        \mathbf{K} =
        \begin{bmatrix}
            k(x_1, x_1) & k(x_1, x_2) \\
            k(x_2, x_1) & k(x_2, x_2) \\
        \end{bmatrix}
    \]

    Because the kernel is positive definite, we know that all the eigenvalues of $\mathbf{K}$ is positive, resulting in

    \[
        \det(\mathbf{K}) = k(x_1, x_1)k(x_2, x_2) - k(x_1, x_2)k(x_2, x_1) > 0
    \]

    Since $k(x_1, x_2) = k(x_2, x_1)$

    \[
        k(x_1, x_1)k(x_2, x_2) - k(x_1, x_2)^2 > 0
    \]

    After simplification, we obtain

    \[
        k(x_1, x_2)^2 < k(x_1, x_1)k(x_2, x_2)
    \]

    Specially, when $x_1 = x_2$, $k(x_1, x_1) = k(x_1, x_2) = k(x_1, x_2)$, resulting in:

    \[
        k(x_1, x_2)^2 = k(x_1, x_1)k(x_2, x_2)
    \]

    Finally, we conclude that:

    \[
        k(x_1, x_2)^2 \le k(x_1, x_1)k(x_2, x_2)
    \]

\end{solution}

\newpage