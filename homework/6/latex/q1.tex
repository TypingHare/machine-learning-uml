\question Q1\droppoints

\begin{solution}
    \text{(a)} The example matrix we discussed in class was
    \[
        A =
        \begin{bmatrix}
            1 & 2 \\
            3 & 4 \\
        \end{bmatrix}
    \]

    To find the eigenvalues of this matrix, we first find

    \[
        A - \lambda I =
        \begin{bmatrix}
            1 & 2 \\
            3 & 4 \\
        \end{bmatrix}
        -
        \lambda
        \begin{bmatrix}
            1 & 0 \\
            0 & 1 \\
        \end{bmatrix}
        =
        \begin{bmatrix}
            1 - \lambda & 2           \\
            3           & 4 - \lambda \\
        \end{bmatrix}
    \]

    Set the determinant of this matrix equal to $0$, we have

    \[
        \det(A - \lambda I)
        = (1 - \lambda)(4 - \lambda) - 3 \times 2
        = \lambda^2 - 5\lambda - 2
        = 0
    \]

    We can use quadratic formula to solve this equation:

    \[
        \lambda
        = \frac{-(-5) \pm \sqrt{(-5)^2 - 4 \times 1 \times (-2)}}{2 \times 1}
        = \frac{5 \pm \sqrt{33}}{2}
    \]

    Finally, we obtain $\lambda_1 = 5.3723$ and $\lambda_2 = -0.3723$.

    \text{(b)} Consider the following matrix:

    \[
        B =
        \begin{bmatrix}
            5 & -2 \\
            1 & 2  \\
        \end{bmatrix}
    \]

    The characteristic equation is given by:

    \[
        \begin{align*}
            \det(B - \lambda I)
            &= (5 - \lambda)(2 - \lambda) - 1 \times (-2) \\
            &= \lambda^2 - 7\lambda + 12 \\
            &= (\lambda - 3)(\lambda -4) \\
            &= 0 \\
        \end{align*}
    \]

    We obtain $\lambda_1 = 3$ and $\lambda_2 = 4$.
    Therefore, the matrix $B$ serves as an example matrix having at least one negative elements but all eigenvalues are positive.
\end{solution}

\newpage