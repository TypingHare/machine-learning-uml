2. Let $P(B = i)$ represent the probability of selecting box $i$, where $i \in \{ r, b, g\}$.
Let $P(F = j)$ represent the probability of selecting fruit $j$, where $j \in \{ a, o, l \}$
Let us set up the probability table that covers all the data:

\begin{table}[h]
    \centering
    \begin{tabular}{ c | c  c  c }
        \hline
        F\backslash B & r   & b   & g   \\
        \hline
        $a$           & $3$ & $1$ & $3$ \\
        $o$           & $4$ & $1$ & $3$ \\
        $l$           & $3$ & $0$ & $4$ \\
        \hline
    \end{tabular}
    \label{tab:q2}
\end{table}

(1) To find the probability of selecting an apple, we use the law of total probability across all boxes:

\[
    P(F = a) = \sum_{i = r, b, g}{P(F = a, B = i)} = \sum_{i = r, b, g}{P(F = a \mid B = i)P(B = i)}
\]

From the table we can find the conditional probabilities:

\begin{align*}
    P(F = a \mid B = r) &= \frac{3}{3 + 4 + 3} = \frac{3}{10} \\
    P(F = a \mid B = b) &= \frac{1}{1 + 1 + 0} = \frac{1}{2} \\
    P(F = a \mid B = g) &= \frac{3}{3 + 3 + 4} = \frac{3}{10}
\end{align*}

Therefore,

\begin{align*}
    P(F = a)
    &= \sum_{i = r, b, g}{P(F = a \mid B = i)P(B = i)} \\
    &= \frac{3}{10} \times 0.2 + \frac{1}{2} \times 0.2 + \frac{3}{10} \times 0.6 \\
    &= \frac{3}{50} + \frac{5}{50} + \frac{9}{50} \\
    &= \frac{17}{50}
\end{align*}

(2) The probability that an orange is selected from a green box is calculated by

\[
    P(B = g \mid F = o)
    = \frac{P(B = g, F = o)}{P(F = o)}
\]

We first find the $P(F = o)$:

\begin{align*}
    P(F = o)
    &= \sum_{i = r, b, g}{P(F = o \mid B = i)P(B = i)} \\
    &= \frac{4}{3 + 4 + 3} \times 0.2 + \frac{1}{1 + 1 + 0} \times 0.2 + \frac{3}{3 + 3 + 4} \times 0.6 \\
    &= \frac{4}{10} \times 0.2 + \frac{1}{2} \times 0.2 + \frac{3}{10} \times 0.6 \\
    &= \frac{4}{50} + \frac{5}{50} + \frac{9}{50} \\
    &= \frac{18}{50}
\end{align*}

Then, we obtain

\[
    P(B = g \mid F = o) = \frac{P(B = g, F = o)}{P(F = o)} = \frac{9}{50} \times \left(\frac{18}{50}\right)^{-1} = \frac{1}{2}
\]

